\documentclass[a4paper,10pt]{article}
\usepackage[utf8]{inputenc}
\usepackage{amsmath}
\usepackage{amssymb}
\usepackage{geometry}
\usepackage[backend=biber]{biblatex} %Imports biblatex package
\addbibresource{bibliography.bib} %Import the bibliography file

%opening
\title{Simulazione}
\author{angelo}

\begin{document}

\maketitle

\begin{abstract}
Simulazione modello di Hopfield variando la distribuzione dei pattern e la temperatura.
\end{abstract}

\section*{Introduzione}

Devo fare la seguente simulazione. Il mio modello è composto da $N$ neuroni $s_i$ (con $i=1\ldots N$) il cui stato in ogni istante è un elemento di $\{-1,+1 \}$, da $p$ pattern $\xi^\mu$ (con $\mu=1\ldots p$) che sono vettori in $\mathbb{R}^N$, estratti secondo una distribuzione di probabilità, e da dei coupling $J_{ij}$ per $i\neq j$ che sono numeri reali e dipendono dai pattern secondo la regola di Hebb:
\begin{equation}
 J_{ij}=\frac{1}{N}\sum_{\mu = 0}^p \xi_i^\mu \xi_j^\mu \ ,\ i\neq j .
 \label{eq:hebb}
\end{equation}
Notiamo che i coupling dipendono dall'estrazione iniziale dei pattern e da lì sono fissati.

L'energia di una configurazione è definita dalla seguente hamiltoniana:
\[
 H(\underline{s}) = -\frac{1}{2} \sum_{i\neq j} J_{ij} s_i s_j\ .
\]
\`{E} un sistema dinamico discreto che si muove secondo una dinamica di Glauber \cite{10.1063/1.1703954} con update sequenziale: in ogni step temporale viene scelto un neurone a caso e con la seguente probabilità cambierà valore (flipperà) o meno:
\[
 P(s_i(t+1)=s) = \frac{1 + s \tanh (\beta \phi_i(\underline{s}(t)))}{2} ,
\]
dove $\phi_i(\underline{s}(t))$ è il campo locale in $i$ al tempo $t$ è definito da:
\[
 \phi_i(\underline{s}(t)) = \sum_{j=1,j\neq i}^N J_{ij}s_j(t) ;
\]
e  $\beta = 1/T$ è l'inverso della temperatura $T$, che è un altro parametro del sistema ed è un numero reale non negativo. In particolare ci interessano i casi di temperatura nulla, vicino allo zero, temperatura critica e vicino alla temperatura critica

\section{Prima simulazione}

Estraggo un sample di $p$ pattern $\xi^\mu$ con $\mu=1\ldots p$, i.i.d. secondo la seguente distribuzione dipendende dal parametro $a$:
\[
 P(\xi^\mu)=\prod_{i=1}^N p(\xi_i^\mu),
\]
\[
 p(\xi_i^\mu) = \frac{a}{2\sqrt{2}} e^{-|\xi|/\sqrt{2}} + \frac{1-a}{2} \left[ \delta(\xi-1) + \delta(\xi+1) \right].
\]
Sulla base di questi pattern sono determinati i coupling $J_{ij}$, inizializzo il sistema sul primo pattern, faccio evolvere il sistema per un tempo $t_{max}$. Per ogni istante di tempo calcolo la magnetizzazione $m_1$ rispetto al primo pattern:
\[
 m_1(t)=\frac{1}{N}\sum_{i=1}^N \xi_i^1 s_i(t) .
\]
Mi stampo e plotto la magnetizzazione in funzione del tempo.

Ripeto per un certo numero di volte da estrazione dei pattern a produzione di magnetizzazione in funzione del tempo, poi faccio la media di $m_1$ rispetto ai vari sample estratti.

\subsection{Input, Output, Execution path}

\textbf{Input}: \begin{itemize}
\item $N$, è un numero intero positivo. Inizialmente provo con 1000; andrebbero fatte almeno due simulazioni con due valori diversi, magari uno il doppio dell'altro per motivi di scaling.
\item $p$, è un numero intero positivo. Inizio con 9 per essere nel limite di carico nullo (è minore di $1\%N=10$) e avere un numero dispari di pattern.
\item $a$, parametro da cui dipende la distribuzione dei pattern, è un numero reale tra 0 e 1.
\item $T$, la temperatura, un numero reale non nullo, ci interessano i valori vicino a 0 e alla temperatura critica.
\item $t_{max}$, per quanti sweep far evolvere il sistema, è un numero intero positivo. Inizio con $100$, ma devo guardare se la magnetizzazione circa smette di cambiare dopo $N$ step, in tal caso posso assumere che il sistema ha termalizzato.
\item $s$, il numero di sample dei pattern e la conseguente simulazione (stabilità primo pattern) da fare, è un numero intero positivo. Inizio con $20$.                                                                                                                                                                                                                                             \end{itemize}


\textbf{Output}:
\begin{itemize}
 %\item $m_1(t)$: magnetizzazione del primo pattern in funzione del tempo per ogni sample, in un formato CSV su due colonne: nella prima un numero progressivo che indica il tempo t, nella seconda un numero reale che indica $m_1(t)$.
 \item $(\overline{m_1(t)}, std(m_1(t)))$: media, rispetto ai vari sample, delle magnetizzazione del primo pattern in funzione del tempo, in un in un formato CSV su due colonne: nella prima un numero reale che indica $\overline{m_1(t)}$, nella seconda un numero reale che indica $std(m_1(t)))$.
\end{itemize}

\textbf{Execution path}
\begin{enumerate}
% \item da input a elenco $m_1(t)$ per sample.
 \item da input a elenco $(\overline{m_1(t)}, std(m_1(t)))$.
% \item da elenco $m_1(t)$ a plot.
 \item da elenco $(\overline{m_1(t)}, std(m_1(t)))$ a plot.
\end{enumerate}

\section{Seconda simulazione draft}

Come la prima ma partendo da una mistura (continua) e calcolando la magnetizzazione rispetto a tale mistura.
Nota: è normale esca la magnetizzazione più grande di 1 perché calcoliamo la magnetizzazione rispetto a cose continue.

\section{Terza simulazione}






\printbibliography

\end{document}
